% Copyright (C)  2014 Richard Bäck.
% Permission is granted to copy, distribute and/or modify this document
% under the terms of the GNU Free Documentation License, Version 1.3 or
% any later version published by the Free Software Foundation; with no
% Invariant Sections, no Front-Cover Texts, and no Back-Cover Texts.  A
% copy of the license is included in the section entitled "GNU Free
% Documentation License".

\documentclass[a4paper,9pt]{article}
\usepackage{amsmath}
\usepackage[utf8]{inputenc}
\usepackage{palatino}
\usepackage{graphicx}
\usepackage{fancyhdr}
\usepackage{eurosym}
\usepackage{hyperref}
\usepackage{multicol}
\usepackage{titlesec}
\usepackage[german]{babel}
\usepackage[margin=1.5cm,vmargin={0pt,1cm}]{geometry}

\DeclareGraphicsExtensions{.pdf}

\setlength{\headheight}{2.5cm}
\setlength{\headsep}{0.5cm}
\setlength{\textheight}{25.2cm}

\pagestyle{fancy}
\lhead{Richard Bäck}
\chead{}
\rhead{\today}
\lfoot{Das Anlagevermögen \& die Anlagenabschreibung}
\cfoot{}
\rfoot{Seite \thepage}
\renewcommand{\headrulewidth}{0.4pt}
\renewcommand{\footrulewidth}{0.4pt}

\newcommand\tkonto[3]{
\begin{tabular}{l|l}
  \multicolumn{2}{c}{#1}\\
  \hline
  #2 & #3
\end{tabular}}

\title{Das Anlagevermögen \& die Anlagenabschreibung}
\author{Richard Bäck}

\begin{document}
\maketitle
\thispagestyle{fancy}

\section{Anlagenaktivierung}
\label{sec:anlagenaktivierung}
\begin{equation*}
  \label{eq:anschaffungsbuchung}
  \begin{array}{l}
    \text{0??? Anlagenkonto}\\
    \text{2500 Vorsteuer}\\
  \end{array}
  \left\slash
  \begin{array}{l}
    \text{33??? Lieferantenkonto (oder 2800 Bank etc.)}\\
  \end{array}
\right.
\end{equation*}
\\
Die Nebenkosten werden gleich verbucht. Bei Preisminderungen ist der
Buchungsatz genau umgedreht.

\section{Abschreibung}
\subsection{Nutzungsdauer}
Die Höhe des Abschreibungsbetrags hängt von der länge der
Nutzungsdauer ab. Die Nutzungsdauer ist für jede Anlage
verschieden. Am besten holt man entweder Rat bei einem Steuerberater
oder von dem Finanzamt eine Tabelle für die Nutzungsdauer
verschiedener Anlagenarten. Den Beginn der Nutzungsdauer markiert die
\textbf{Inbetriebnahme} der Anlage.
\begin{equation}
  Abschreibungsbetrag = { Anschaffungswert \over Nutzungsdauer }
\end{equation}

\subsection{Halbjahresregel}
Wird die Anlage im ersten Halbjahr in Betrieb genommen, wird für
dieses Jahr der volle Betrag abgeschrieben. Ist die Inbetriebnahme
erst im zweiten Halbjahr, wird nur der halbe Betrag für dieses und
folglich auch für das letzte Jahr abgeschrieben.

\subsection{Lineare Abschreibung}
Es wird jährlich der Abschreibungsbetrag von der Anlage
abgebucht. Wenn die Anlage am Ende der Nutzungsdauer im Betrieb
bleibt, wird Abschreibungsbetrag - 1 abgebucht um den
,,Erinnerungseuro'' am Anlagenkonto zu erhalten.

\subsection{Substanzabschreibung}
Bei förderbaren Grundstücken verliert auch das Grundstück mit der
geförderten Substanz an Wert (= ein Vorkommen wird weniger, weil es
abgebaut wird). Man benötigt für die Abschreibung eine Schätzung der
abbaufähigen Substanz.

\begin{equation}
  Abschreibungsbetrag = { \textit{Anschaffungswert} \over
\textit{abbaufähige Substanz} } \cdot \textit{Fördermenge}
\end{equation}

\section{Anlagenbuchführung}
Im Anlagenverzeichnis (händische Aufzeichnung) bzw. in der
Anlagendatei (elektronische Aufzeichnung) werden die Anlagen, ihr
Anschaffungswert, die Abschreibungsbeträge und die folglichen
Buchwerte erfasst. Die Führung dieses Buches ist vom Steuerrecht
vorgeschrieben.

\section{Verbuchung der Abschreibung}
\subsection{Direkte Abschreibung}
Der Abschreibungsbetrag wird direkt von der Anlage abgebucht.

\begin{equation*}
  \begin{array}{l}
    \text{7010 Abschreibungen von Sachanlagen}\\
  \end{array}
  \left\slash
    \begin{array}{l}
      \text{0??? Anlagenkonto}\\
    \end{array}
  \right.
\end{equation*}

\subsection{Indirekte Abschreibung}
Der Abschreibungsbetrag wird auf ein Stellvertreterkonto des
Anlagenkontos gebucht und auf die Passiva abgeschlossen (= der
Anlagenwert wird indirekt vermindert).

\begin{equation*}
  \begin{array}{l}
    \text{7010 Abschreibungen von Sachanlagen}\\
  \end{array}
  \left\slash
    \begin{array}{l}
      \text{0??? Kumulierte Abschreibungen zu 0???}\\
    \end{array}
  \right.
\end{equation*}

\subsection{Geringwertige Wirtschaftsgüter}
Anlagen mit einem Anschaffungswert von weniger oder gleich \euro{}
400,- (exkl. USt.) können nach dem Kauf sofort abgeschrieben werden.

\begin{equation*}
  \begin{array}{l}
    \text{7030 Abschreibung geringwertiger Wirtschaftsgüter}\\
    \text{2500 Vorsteuer}\\
  \end{array}
  \left\slash
    \begin{array}{l}
      \text{33??? Lieferantenkonto (oder 2800 Bank etc.)}\\
    \end{array}
  \right.
\end{equation*}

Es ist auch möglich die Anlage, wie die normalen, erst am Jahresende
abzuschreiben:
\begin{itemize}
\item { Sofortige Buchung beim Kauf:\\
    \begin{equation*}
      \begin{array}{l}
        \text{0??? Geringwertige ...}\\
        \text{2500 Vorsteuer}\\
      \end{array}
      \left\slash
        \begin{array}{l}
          \text{33??? Lieferantenkonto (oder 2800 Bank etc.)}\\
        \end{array}
      \right.
    \end{equation*}
  }
\item { Buchung am Jahresende:\\
    \begin{equation*}
      \begin{array}{l}
        \text{7030 Abschreibung geringwertiger Wirtschaftsgüter}\\
      \end{array}
      \left\slash
        \begin{array}{l}
          \text{0??? Geringwertige ...}\\
        \end{array}
      \right.
    \end{equation*}
  }
\end{itemize}

\section{Andere Buchungen}
\subsection{Selbst erstellte Anlagen}
Anstatt der Anschaffungskosten, werden die Herstellkosten aktiviert.
\begin{equation*}
  \begin{array}{l}
    \text{0??? Anlagenkonto}\\
  \end{array}
  \left\slash
    \begin{array}{l}
      \text{4850 Aktivierte Eigenleistungen}\\
    \end{array}
  \right.
\end{equation*}

\subsection{In Bau befindliche Anlagen}
Wenn eine Anlage über das Geschäftsjahr hinweg weiter gebaut wird,
greift dieses Thema. In Bau befindliche Anlagen können
\textbf{abgeschrieben} werden.\\
\\
Buchung einer Teilzahlung:
\begin{equation*}
  \begin{array}{l}
    \text{0710 Anlagen im Bau}\\
    \text{2500 Vorsteuer}\\
  \end{array}
  \left\slash
    \begin{array}{l}
      \text{4850 Aktivierte Eigenleistungen}\\
    \end{array}
  \right.
\end{equation*}
\\
Buchung bei selbst erstellten Teilen:
\begin{equation*}
  \begin{array}{l}
    \text{0710 Anlagen im Bau}\\
  \end{array}
  \left\slash
    \begin{array}{l}
      \text{33??? Lieferantenkonto (oder 2800 Bank etc.)}\\
    \end{array}
  \right.
\end{equation*}\\
\\
Nach der Fertigstellung:
\begin{equation*}
  \begin{array}{l}
    \text{0??? Anlagenkonto}\\
  \end{array}
  \left\slash
    \begin{array}{l}
      \text{0710 Anlagen im Bau}\\
    \end{array}
  \right.
\end{equation*}

\section{Instandhaltung von Anlagen}
Man spricht von einer Instandhaltung, wenn:
\begin{itemize}
  \item der ordnungsgemäße Zustand erhalten wird
  \item die Wesensart der Anlage nicht verändert wird
\end{itemize}

Die Kosten werden so verbucht:
\begin{equation*}
  \begin{array}{l}
    \text{7200 Instandhaltung durch
Dritte}\\
     \text{2500 Vorsteuer}\\
  \end{array}
  \left\slash
    \begin{array}{l}
      \text{33??? Lieferantenkonto (oder 2800 Bank etc.)}\\
    \end{array}
  \right.
\end{equation*}

\section{Erweiterung einer Anlage}
Von einer Erweiterung spricht man, wenn:
\begin{itemize}
  \item die Anlage durch eine Aufwendung in seiner Substanz vermehrt wird
  \item die Gebrauchsmöglichkeit einer Anlage wird wesentlich verändert
\end{itemize}

Es wird die Erweiterung aktiviert, wie
in~\autoref{sec:anlagenaktivierung} -~\nameref{sec:anlagenaktivierung}
und ein neuer Abschreibungsbetrag berechnet.

\begin{equation}
  { {Anschaffungskosten + Erweiterungskosten } \over Restnutzungsdauer
  } = neuer Abschreibungsbetrag
\end{equation}

\section{Ausscheiden einer Anlage}
\subsection{Ausscheidung bei einem Verkauf}
\label{subsec:verkauf}
Zuerst wird der Erlös verbucht. Danach wird die Anlagen
ausgebucht. Bei einer Kapitalgesellschaft wird bei den
Abschlussarbeiten saldiert: $ \textit{Verkaufserlöse} - Buchwerte = Saldo$. Bei
$\geq 0$/<0 wandern die Erlöse und die Buchwerte auf ein eigenes
Ertragskonto/Verlustkonto.

Erlösbuchung:
\begin{equation*}
  \begin{array}{l}
    \text{20??? Forderungskonto (oder 2800 Bank etc.)}\\
  \end{array}
  \left\slash
    \begin{array}{l}
      \text{4600 Erlöse aus dem Abgang von Anlagen}\\
      \text{3500 Umsatzsteuer}\\
    \end{array}
  \right.
\end{equation*}

Ausbuchung der Anlage:
\begin{equation*}
  \begin{array}{l}
    \text{7820 Buchwert abgegangener Anlagen}\\
  \end{array}
  \left\slash
    \begin{array}{l}
      \text{0??? Anlagenkonto}\\
    \end{array}
  \right.
\end{equation*}

Saldo $\geq 0$:
Umbuchung des Verkaufserlöses:
\begin{equation*}
  \begin{array}{l}
    \text{4600 Erlöse aus dem Abgang von Anlagen}\\
  \end{array}
  \left\slash
    \begin{array}{l}
      \text{4630 Erträge aus dem Abgang von Anlagen}\\
    \end{array}
  \right.
\end{equation*}\\
\\
Umbuchung des Buchwertes:
\begin{equation*}
  \begin{array}{l}
    \text{4630 Erträge aus dem Abgang von Anlagen}\\
  \end{array}
  \left\slash
    \begin{array}{l}
      \text{7820 Buchwert abgegangener Anlagen}\\
    \end{array}
  \right.
\end{equation*}

Saldo <0:
Umbuchung des Verkaufserlöses:
\begin{equation*}
  \begin{array}{l}
    \text{4600 Erlöse aus dem Abgang von Anlagen}\\
  \end{array}
  \left\slash
    \begin{array}{l}
      \text{7830 Verluste aus dem Abgang von Anlagen}\\
    \end{array}
  \right.
\end{equation*}\\
\\
Umbuchung des Buchwertes:
\begin{equation*}
  \begin{array}{l}
    \text{7830 Verluste aus dem Abgang von Anlagen}\\
  \end{array}
  \left\slash
    \begin{array}{l}
      \text{7820 Buchwert abgegangener Anlagen}\\
    \end{array}
  \right.
\end{equation*}

\subsection{Ausscheidung bei einem Schadensfall}
Erlös aus einem etwaigen Versicherungbetrag:
\begin{equation*}
  \begin{array}{l}
    \text{2800 Bank (2700 Kassa etc.)}\\
  \end{array}
  \left\slash
    \begin{array}{l}
      \text{4610 Versicherungsentschädigungen für Anlagenabgänge}\\
    \end{array}
  \right.
\end{equation*}\\
\\
Die Ausbuchung und die Saldierungsbuchung ist gleich wie
in~\autoref{subsec:verkauf} -~\nameref{subsec:verkauf}. Der
Unterschied liegt nur darin, dass das Konto \textit{7820 Buchwert
abgegangener Anlagen} zu \textit{7819 Sonstige Schadensfälle} wird.

\subsection{Ausscheidung nach voller Abschreibung}
Wird eine vollabgeschriebene Anlage aus dem Betrieb entfernt, so wird
der Erinnerungseuro ausgebucht.

\begin{equation*}
  \begin{array}{l}
    \text{7010 Abschreibungen von Sachanlagen}\\
  \end{array}
  \left\slash
    \begin{array}{l}
      \text{0??? Anlagenkonto 1,00}\\
    \end{array}
  \right.
\end{equation*}


\end{document}
