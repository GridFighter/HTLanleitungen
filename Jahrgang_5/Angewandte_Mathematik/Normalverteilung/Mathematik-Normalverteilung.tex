\documentclass[a4paper,10pt]{article}
\usepackage[utf8]{inputenc} % enable umlauts in the source files
\usepackage{palatino}
\usepackage{graphicx} % use of graphics
\usepackage{amsmath} % for equotation*
\usepackage{eurosym} % the euro symbol
\usepackage{fixltx2e} % super-/subscript
\usepackage{listings} % code listings
\usepackage{fancyhdr} % headers
\usepackage{hyperref} % to create references to chapters etc.
\usepackage[version=3]{mhchem} % chemical formulas
\usepackage[german]{babel} % use german headings
\usepackage[margin=1.5cm,vmargin={0pt,1cm}]{geometry}

\DeclareGraphicsExtensions{.pdf}

\setlength{\headheight}{2.5cm}
\setlength{\headsep}{0.5cm}
\setlength{\textheight}{26cm}

\pagestyle{fancy}
\lhead{Richard Bäck}
\chead{}
\rhead{\today}
\lfoot{}
\cfoot{}
\rfoot{Seite \thepage}
\renewcommand{\headrulewidth}{0.4pt}
\renewcommand{\footrulewidth}{0.4pt}

\title{Zusammenschrift zur Normalverteilung}

\begin{document}
\maketitle
\thispagestyle{fancy}

\section{Normalverteilung}
\label{sec:normalverteilung}
\subsection{Bestimmte Wahrscheinlichkeit}
\begin{equation}
  dnorm(x, \mu, \sigma) = \text{Wahrscheinlichkeit für einen bestimmten Wert}
\end{equation}

\subsection{Kumulative Wahrscheinlichkeit}
\begin{equation}
  pnorm(x, \mu, \sigma) = \text{Wahrscheinlichkeit für höchstens einen Wert}
\end{equation}

\subsection{Inverse kumulative Wahrscheinlichkeit}
\begin{equation}
  qnorm(p, 0, 1) = x = \text{Wert für eine bestimmte Wahrscheinlichkeit}
\end{equation}

\section{Verschiebung zur Standardnormalverteilung}
\begin{equation}
  u = \frac{x - \mu}{\sigma} = qnorm(p, 0, 1)
\end{equation}

\section{Normalverteilung und Stichproben}
Wenn eine Stichprobe mit n Werten von einer Grundgesamtheit genommen
wird, dann verändern sich $ \mu $ und $ \sigma $, die eigentlich für
die Grundgesamtheit gelten, folgendermaßen:
\begin{equation}
  \overline{\mu} = \mu
\end{equation}
\begin{equation}
  \overline{\sigma} = \frac{\sigma}{\sqrt{n}}
\end{equation}

\section{Annäherung der Binomialverteilung zur Normalverteilung}
Es gilt:
\begin{equation}
  \mu = n * p
\end{equation}
\begin{equation}
  \sigma = \sqrt{n * p * (1 - p)}
\end{equation}
\\
Es muss erkannt werden, dass die gegebenen Daten binomialverteilt
sind! Dies ist gegeben, wenn die Daten immer nur Ja/Nein Ergebnisse
sind und bei einer Wiederholung sich die Wahrscheinlichkeit nicht
ändert.
\section{Suchen nach Variablen anhand einer Binomialverteilung}
\subsection{Grundlegendes}
\label{sec:variabliensuchebinomialverteilunggrundlegendes}
Mit den im Kapitel ~\autoref{sec:normalverteilung}
-~\nameref{sec:normalverteilung} beschriebenen Funktionen können nun
folgende Werte gesucht werden:
\begin{description}
\item[x] Der bestimmte Wert von einer Grundgesamtheit
\item[n] Die Grundgesamtheit
\item[p] Die Wahrscheinlichkeit, wieviel von der Grundgesamtheit y
  eintritt ($ y = n * p $)
\item[p\textsubscript{2}] Die Wahrscheinlichkeit einen Wert x von der
  Grundgesamtheit n zu erhalten
\end{description}
Wichtig ist, dass die folgenden Vorgehensweisen sich nicht auf die
Binomialverteilung beschränkt. Es muss nur $ \mu $ und $ \sigma $
gegeben sein! Für all die nachstehenden Suchen sind diese beide
definiert durch:
\begin{itemize}
\item $\mu = p * n $
\item $ \sigma = \sqrt{n * p * (1  - p)} $
\end{itemize}

\subsection{Suche nach x}
Gegebene Variablen:
\begin{itemize}
\item n
\item p
\item p\textsubscript{2}
\end{itemize}
\begin{equation}
  qnorm(p_2, 1, 0) = \frac{x -\mu}{\sigma} \text{solve, x} \rightarrow
\end{equation}

\subsection{Suche nach n}
Gegebene Variablen:
\begin{itemize}
\item x
\item p
\item p\textsubscript{2}
\end{itemize}
\begin{equation}
  qnorm(p_2, 0, 1) = \frac{x -\mu}{\sigma} \text{solve, n} \rightarrow
\end{equation}

\subsection{Suche nach p}
Gegebene Variablen:
\begin{itemize}
\item x
\item n
\end{itemize}
Wenn es sich um eine Binomialverteilung handelt, dann gilt: $ \sigma =
\sigma_0 $, da eine Wurzel gezogen wurde.
\begin{equation}
  pnorm(x, \mu, \sigma) = p_2
\end{equation}

\section{Zweiseitiger Zufallsstreubereich}
\subsection{Grundlegendes}
Beim zweiseitigen Zufallsstreubereich wird ermittelt, welche Unter-
und Obergrenze die Werte in einem bestimmten Bereich symmetrisch
verteilt um den Erwartungswert besitzt.
\\
Gegebene Variablen:
\begin{description}
\item[$ \mu $]
\item[$ \sigma $]
\item[p] Die Wahrscheinlichkeit (Größe) des Streubereichs symmetrisch
  um den Erwartungswert $ \mu $
\end{description}

\subsection{Berechnung von x\textsubscript{u,o}}
\begin{equation}
  \alpha = 1 - p
\end{equation}
\begin{equation}
  x_u = \mu - qnorm(1 - \frac{\alpha}{2}, 0, 1) * \sigma
\end{equation}
\begin{equation}
  x_o = \mu + qnorm(1 - \frac{\alpha}{2}, 0, 1) * \sigma
\end{equation}

\subsection{Berechnung von x\textsubscript{u,o} bei Stichproben}
Es kommt nun zustätzlich die gegebene Variable der Stichprobenanzahl
\textbf{n} hinzu.
\begin{equation}
  \alpha = 1 - p
\end{equation}
\begin{equation}
  \overline{\sigma} = \frac{\sigma}{\sqrt{n}}
\end{equation}
\begin{equation}
  x_u = \mu - qnorm(1 - \frac{\alpha}{2}, 0, 1) * \overline{\sigma}
\end{equation}
\begin{equation}
  x_o = \mu + qnorm(1 - \frac{\alpha}{2}, 0, 1) * \overline{\sigma}
\end{equation}


\section{Konfidenzintervall}
\label{sec:konfidenzintervall}
\subsection{Grundlegendes}
Das Konfidenzintervall gibt an, in welchen Intervall die Standardabweichung am
ehestens bei gegebenen Stichproben liegt. Das bedeutet, dass $ \sigma $ selbst
nicht ermittelt werden kann, jedoch der Bereich, in der sich $ \sigma $ bewegt.

Gegeben Variablen müssen sein:
\begin{description}
\item[m] Ein einzeiliger Stichprobenvektor
\item[s] Die Standardabweichung s berechnet aus m errechnet mit $
  Stdev() $ bzw. $ stdev() $
\item[p] Das Konfidenznevau (meistens 95 \%)
\end{description}

\subsection{Suche nach dem Konfidenzintervall}
Es werden folgende Variablen gesucht:
\begin{description}
\item[$ \sigma_u $] Die untere Grenze des Vertrauensbereichs
\item[$ \sigma_o $] Die obere Grenze des Vertrauensbereichs
\end{description}

Berechnung:
\begin{equation}
  \alpha = 1 - p
\end{equation}
\begin{equation}
  p_u = 1 - \frac{\alpha}{2}
\end{equation}
\begin{equation}
  p_o = \frac{\alpha}{2}
\end{equation}
\begin{equation}
  \sigma_u = s * \sqrt{\frac{f}{qchisq(p_u , f)}}
\end{equation}
\begin{equation}
  \sigma_o = s * \sqrt{\frac{f}{qchisq(p_o , f)}}
\end{equation}

Der p \%ige Vertrauensbereich für $
\sigma $ liegt dann bei: $ [
\sigma_u ; \sigma_o] $

\section{Ermittlung von $ \mu $}
\subsection{Ermittlung basierend auf Stichproben}
\label{sec:ermittlungvonmuermittlungbasierendaufeinergrundmenge}
Es wird, ähnlich wie bei ~\autoref{sec:konfidenzintervall}
-~\nameref{sec:konfidenzintervall}, die Erwartungswert $ \mu $ zu Stichproben in
einem Intervall errechnet. Es ist dabei $ \sigma $ gegeben!
\\\\
Gegebene Variablen:
\begin{description}
\item[m] Ein einzeiliger Stichprobenvektor
\item[n] Die Länge von m
\item[$ \sigma $] Die Standardabweichung
\item[$ \alpha $] Die Signifikanz in Prozent
\end{description}

\subsubsection{Berechnung von $ \mu $}
\begin{equation}
  \overline{x} = mean(m)
\end{equation}
\begin{equation}
  u = qnorm(1 - \frac{\alpha}{2}, 0, 1)
\end{equation}
\begin{equation}
  \mu_u = \overline{x} - u * \frac{\sigma}{\sqrt{n}}
\end{equation}
\begin{equation}
  \mu_o = \overline{x} + u * \frac{\sigma}{\sqrt{n}}
\end{equation}

\subsection{Studentsche Verteilung}
\subsubsection{Grundlegendes}
Es wird der Erwartungswert $ \mu $ auf Grund einer Stichprobe ermittelt. Hier
ist jedoch $ \sigma $ nicht bekannt und muss selbst errechnet werden. Der Rest
deckt sich mit Kapitel
~\autoref{sec:ermittlungvonmuermittlungbasierendaufeinergrundmenge}
-~\nameref{sec:ermittlungvonmuermittlungbasierendaufeinergrundmenge}.
\\
Gegebene Variablen:
\begin{description}
\item[m] Ein einzeiliger Stichprobenvektor
\item[n] Die Länge von m
\item[s] Die Standardabweichung s berechnet aus m errechnet mit $
  Stdev() $ bzw. $ stdev() $
\item[$ \alpha $] Die Signifikanz in Prozent
\end{description}

\subsubsection{Berechnung von $ \mu $}
\begin{equation}
  \overline{x} = mean(m)
\end{equation}
\begin{equation}
  s = Stdev(m) bzw. stdev(m)
\end{equation}
\begin{equation}
  f = n -1
\end{equation}
\begin{equation}
  \mu_u = \overline{x} - qt(1 - \frac{\alpha}{2}, f) * \frac{s}{\sqrt{n}}
\end{equation}
\begin{equation}
  \mu_o = \overline{x} + qt(1 - \frac{\alpha}{2}, f) * \frac{s}{\sqrt{n}}
\end{equation}

\section{$ \chi ^2 $-Verteilung}
\subsection{Grundlegendes}
\label{sec:chiquadratverteilunggrundlegendes}
Die $ \chi ^2 $-Verteilung wird verwendet um zu Überpüfen:
\begin{itemize}
\item Wie wahrscheinlich ist eine bestimmte Standardabweichung s?
  (Kapitel ~\autoref{sec:chisuchenachp} -~\nameref{sec:chisuchenachp})
\item Zu einer Wahrscheinlichkeit p wird welche Standardabweichung s
  erwartet? (Kapitel ~\autoref{sec:chisuchenachs}
  -~\nameref{sec:chisuchenachs})
\item Eine alternierende Standardabweichung s ist zur
  Standardabweichung $ \sigma $ gesucht.
\end{itemize}

Wichtig dabei ist, dass es sich bei der Standardabweichung s um eine
Standardabweichung gegeben von bestimmten Werten handelt. Somit wird s
entweder durch einen externen Rechner gegeben oder selbst mit den
folgenden Funktionen ermittelt:
\begin{description}
\item[$ Stdev(m) $] m ist ein einzeiliger Vektor mit der Grundgesamtheit
\item[$ stdev(m) $] m ist ein einzeiliger Vektor mit Stichproben
\end{description}

\subsection{Freiheitsgrade}
Die Werte n, die freigewählt werden können. Die Definition der Freiheitsgrade:
\begin{equation}
  f = n - 1
\end{equation}

\subsection{Bestimmte Wahrscheinlichkeitsdichte}
\label{sec:bestimmtewahrscheinlichkeitsdichte}
\begin{equation}
  dchisq(x, f) = \text{Wahrscheinlichkeitsdichte für einen bestimmten Wert}
\end{equation}

\subsection{Kumulative Wahrscheinlichkeitsdichte}
\label{sec:kumulativewahrscheinlichkeitsdichte}
\begin{equation}
    x_{\text{prüf}} = f * \frac{s^2}{\sigma^2}
\end{equation}
\begin{equation}
  pchisq(x_{\text{prüf}}, f) = \text{Wahrscheinlichkeitsdichte für höchstens einen Wert}
\end{equation}

\subsection{Inverse kumulative Wahrscheinlichkeitsdichte}
\begin{equation}
  \chi^2_{f,p} = qchisq(p, f) = x_{\text{prüf}}
\end{equation}

\subsection{Suche nach Werten}
Es müssen wie im Kapitel
~\autoref{sec:variabliensuchebinomialverteilunggrundlegendes}
-~\nameref{sec:variabliensuchebinomialverteilunggrundlegendes} $ \mu $
und $ \sigma $ gegeben sein. Außerdem müssen die Freiheitsgrade f
bekannt sein ($ \rightarrow $ also die Anzahl der Werte!).

\subsubsection{Suche nach s}
\label{sec:chisuchenachs}
Gegebene Variablen:
\begin{itemize}
\item p
\item Es ist bekannt ob eine maximale (Kapitel
~\autoref{sec:kumulativewahrscheinlichkeitsdichte}
-~\nameref{sec:kumulativewahrscheinlichkeitsdichte}) oder eine
bestimmte (Kapitel ~\autoref{sec:bestimmtewahrscheinlichkeitsdichte}
-~\nameref{sec:bestimmtewahrscheinlichkeitsdichte}) Standardabweichung
gesucht ist. (Matchad beherrscht nur die inverse kumulative
Wahrscheinlichkeitsdichte!)
\end{itemize}

\begin{equation}
  x_{\text{prüf}} = qchisq(p, f)
\end{equation}
\begin{equation}
  x_{\text{prüf}} = f * \frac{s^2}{\sigma^2} \text{solve, s} \rightarrow
\end{equation}

\subsubsection{Suche nach p}
\label{sec:chisuchenachp}
Gegebene Variablen:
\begin{itemize}
\item s bzw. m für $ Stdev(m) $ oder $ stdev(m) $
\end{itemize}

Es wird somit die Berechnung des Kapitels
~\autoref{sec:kumulativewahrscheinlichkeitsdichte}
-~\nameref{sec:kumulativewahrscheinlichkeitsdichte}
verwendet.

\end{document}
