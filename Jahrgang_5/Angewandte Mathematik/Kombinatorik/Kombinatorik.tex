% Copyright (C)  2015 Richard Bäck.
% Permission is granted to copy, distribute and/or modify this document
% under the terms of the GNU Free Documentation License, Version 1.3 or
% any later version published by the Free Software Foundation; with no
% Invariant Sections, no Front-Cover Texts, and no Back-Cover Texts.  A
% copy of the license is included in the section entitled "GNU Free
% Documentation License".

\documentclass[a4paper,10pt]{article}
\usepackage[utf8]{inputenc} % enable umlauts in the source files
\usepackage{palatino}
\usepackage{graphicx} % use of graphics
\usepackage{amsmath} % for equotation*
\usepackage{eurosym} % the euro symbol
\usepackage{fixltx2e} % super-/subscript
\usepackage{listings} % code listings
\usepackage{fancyhdr} % headers
\usepackage{hyperref} % to create references to chapters etc.
\usepackage[version=3]{mhchem} % chemical formulas
\usepackage[german]{babel} % use german headings
\usepackage[margin=1.5cm,vmargin={0pt,1cm}]{geometry}

\DeclareGraphicsExtensions{.pdf}

\setlength{\headheight}{2.5cm}
\setlength{\headsep}{0.5cm}
\setlength{\textheight}{26cm}

\pagestyle{fancy}
\lhead{Richard Bäck}
\chead{}
\rhead{\today}
\lfoot{}
\cfoot{}
\rfoot{Seite \thepage}
\renewcommand{\headrulewidth}{0.4pt}
\renewcommand{\footrulewidth}{0.4pt}

\title{Zusammenschrift zur Kombinatorik}

\begin{document}
\maketitle
\thispagestyle{fancy}

\section{Permutation}
\subsection{ohne Wiederholung}
\subsubsection{Definition}
Eine Permutation ohne Wiederholung ist der Austausch von \bf{n}
verschiedenen Dingen untereinander. Also können \bf{n} Dinge nacheinander
in einer beliebigen Reihenfolge angeordnet werden.

\begin{equation}
  \label{eq:1}
  n!
\end{equation}

\subsubsection{Mathcad}
In Mathcad gibt es für die Permutation ohne Wiederholung keinen Befehl.

\subsubsection{Beispiel}
\begin{itemize}
\item {
    10 Leute gehen nacheinander durch eine Türe. Anzahl von Möglichkeiten,
    wie sie durch die Türe gehen können:
    \begin{equation}
      \label{eq:3}
      10! = 3628800
    \end{equation}
  }
\item {
    Ein Alphabet besteht aus den Buchstaben 'ABC'. Wieviele Wörter
    können entstehen, wenn keine Wiederholungen enthalten sein dürfen?
    \begin{equation}
      \label{eq:4}
      3! = 6
    \end{equation}
  }
\end{itemize}

\subsection{mit Wiederholung}
\subsubsection{Defintion}
Eine Permutation mit Wiederholung ist der Austausch von \bf{n}
verschiedenen Dingen untereinander. Dabei können aber \bf{m} verschiedene
Teilmengen von *n* doppelt auftreten. *n\textsubscript{m}* stellt die
Anzahl der *m*-ten Teilmenge.

\begin{equation}
  \label{eq:2}
  \frac{n!}{n_0! \cdot n_1! \cdot n_m!}
\end{equation}

\subsubsection{Mathcad}
In Mathcad gibt es für die Permutation mit Wiederholung keinen Befehl.

\subsubsection{Beispiel}
\begin{itemize}
\item {
    Ein Alphabet besteht aus den Buchstaben 'ABBCC'. Wieviele Wörter
    können entstehen?
    \begin{equation}
      \label{eq:5}
      \frac{4!}{2! \cdot 2!} = 6
    \end{equation}
  }
\end{itemize}

\section{Variation}
\subsection{ohne Wiederholung}
\subsubsection{Defintion}
Eine Variation ist eine Auswahl *k* von *n* verschiedenen Elementen
(es gilt $ k <= n $). Dabei ist die Reihenfolge von Bedeutung. In den
*n* Elementen darf jedes Element nur einmal vorkommen.

\begin{equation}
  \label{eq:6}
  \frac{n!}{(n - k)!}
\end{equation}

\subsubsection{Mathcad}
In Mathcad lautet der Befehl:
\begin{equation}
  \label{eq:7}
  permut(n, k)
\end{equation}

\subsubsection{Beispiele}
\begin{itemize}
\item {
    Wieviele zweiziffrige Zahlen kann man aus den Ziffern 1, 4, 6, 8
    und 9 bilden, wenn in der Zahl jede Ziffer nur einmal vorkommen
    darf?
    \begin{equation}
      \label{eq:8}
      permut(5, 2) = \frac{5!}{(5 - 2)!} = 20
    \end{equation}
  }
\end{itemize}

\subsection{mit Wiederholung}
\subsection{Defintion}
Eine Variation mit Wiederholung ist eine Auswahl *k* von *n*
verschiedenen Elementen. Dabei ist die Reihenfolge wichtig, jedoch
können alle ausgewählten *k* Elemente gleich sein.

\begin{equation}
  \label{eq:11}
  n^k
\end{equation}

\subsubsection{Mathcad}
In Mathcad gibt es für die Varation mit Wiederholung keinen Befehl.

\subsection{Beispiele}
\begin{itemize}
\item {
    Ein Alphabet besteht aus den Buchstaben ``ABCDEFG''. Wieviele
    Passwörter können aus diesem Alphabet generiert werden, wenn ein
    Passwort eine länge von 5 Buchstaben besitzen muss?
    \begin{equation}
      \label{eq:12}
      7^5 = 16807
    \end{equation}
  }
\end{itemize}

\section{Kombination}
\subsection{ohne Wiederholung}
\subsubsection{Defintion}
Bei einer Kombination ohne Wiederholung wird eine Auswahl *k* von *n*
Elementen getroffen. Dabei ist die Reihenfolge egal und jedes Element
von *n* kann nur einmal ausgewählt werden.

\begin{equation}
  \frac{n!}{k! \cdot (n - k)!}
\end{equation}

\subsubsection{Mathcad}
\begin{equation}
  \label{eq:14}
  combin(n, k)
\end{equation}

\subsubsection{Beispiele}
\begin{itemize}
\item {
    In einem Koordinatensystem gibt es 10 Punkte. Wieviele Geraden
    können gezeichnet werden, wenn maximal 3 Punkte auf einer Gerade
    liegen dürfen?
    \begin{equation}
      \label{eq:13}
      combin(10, 3) = \frac{10!}{3! \cdot (10 - 3)!} = 120
    \end{equation}
  }
\end{itemize}

\subsection{mit Wiederholung}
\subsubsection{Defintion}
Bei einer Kombination mit Wiederholung wird eine Auswahl *k* von *n*
Elementen getroffen. Dabei ist die Reihenfolge egal und es können alle
ausgewählten *k* Elemente gleich sein.

\begin{equation}
  \label{eq:15}
  \binom{n + k - 1}{k}
\end{equation}

\subsubsection{Mathcad}
In Mathcad gibt es für die Kombination mit Wiederholung keinen Befehl,
jedoch kann dafür der $combin()$-Befehl missbraucht werden.
\begin{equation}
  \label{eq:16}
  combin(n + k - 1, k)
\end{equation}

\subsubsection{Beispiele}
\begin{itemize}
\item {
    In einer Klasse mit 20 Schülern soll die Aufgabe des
    Klassenordners und des Kassiers bestimmt werden, dabei kann beide
    Funktionen auch eine Person übernehmen. Wieviele Möglichkeiten
    gibt es?
    \begin{equation}
      \label{eq:17}
      combin(20 + 2 - 1, 2) = \binom{20 + 2 - 1}{2} = 210
    \end{equation}
  }
\end{itemize}

\section{Wahrscheinlichkeit}
\subsection{Definition}
Die Wahrscheinlichkeit zu berechnen, wenn die Anzahl *x* von
Möglichkeiten gegeben ist, ist relativ einfach:

\begin{equation}
  \label{eq:9}
  \frac{1}{x}
\end{equation}

\subsection{Beispiel}
\begin{itemize}
\item {
    Wie wahrscheinlich ist es in Lotto 6 aus 45 zu gewinnnen den
    Jackpot zu gewinnen (6 richtige Zahlen benötigt, egal in welcher
    Reihenfolge)?
    \begin{equation}
      \label{eq:10}
      \frac{1}{combin(45, 6)} = \frac{1}{\frac{45!}{6! \cdot (45 - 6)!}} = \frac{1}{8145060}
    \end{equation}
  }
\end{itemize}


\end{document}